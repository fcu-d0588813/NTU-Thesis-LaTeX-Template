% !TeX root = ../main.tex

\chapter{諸論}

在升學階段,許多大學開放學測個人申請、甄試入學申請等等的申請入學方式,在推甄名額越來越多,考試名額越來越少的情況下,學校越來越傾向以書面審查當作初試的審查標準。然而,現在的選才,除了課業表現外,社團活動、學習特質、動機與能力更是重要,學校希望能看到學生的更多面向,而不單單只是成績上的數字,審查資料時,更需要看見學生自發性學習的精神、學習動機和人格特質,書面審查資料屬於半結構化資料,包含了結構化資料的部分如:姓名、生日、住址等等,以及非結構化資料如:自傳、就讀動機等等,由於每個學生的書面資料格式不同,審查委員難以有效地找到資料中重要的訊息。

審查委員依據尺規評估學生的審查資料是否適合本科系,審查資料包含了學生的基本資料、就讀高中、成績、技能、專長、興趣、比賽經驗、幹部經驗、社團活動、自傳、讀書計畫、就讀動機等資料,學校提供一個尺規,其中包含了學生的一般學習表現、資工相關學習表現、多元表現、自我學習能力,每一項有1~5級的評分,其中一般學習表現根據學生修習數學、物理與英文課程的表現來評分,表現分別分為傑出、優秀、中上表現、中等表現及表現不佳。多元表現根據參加程式相關的競賽或課程,參與國際性、地區性或是校內性的數學、物理或是資訊科技等比賽,以及相關程式碼作品的表現及社團幹部經驗去評分。自我學習能力則是依照讀書計畫的縝密度以及就讀動機去評估學生。審查委員會根據這個尺規對書面資料打分數,最後以這個標準去評估學生是否可以錄取,然而在審查學生的書面資料過程中,可能會面臨:(1)審查委員分配到一部份的書面資料,由於各個審查委員的主觀意識不同,導致評分標準有落差,對審查有失公允,且對學生也不公平。(2)審查資料眾多,加上資料格式繁雜,審查委員無法仔細審查資料。(3)尺規的定義模糊,造成審查委員難以判斷(4)人工審查費時費力為了減輕審查委員的負擔,在眾多格式繁雜的書審資料中找到學生的特質,所以本研究提出自動評估學生審查資料的方法,希望能透過深度學習的方法讓機器能夠自動配對審查資料中的文字到尺規,並且得出分數,因此,我們提出了自動評估學生審查資料的方法,能夠幫助審查委員在審查書面資料上的效率與公平性,從大篇幅的書面資料中,提取書面資料中有用的資訊如:學生的競賽表現、學業成績、幹部經驗與就讀動機等等,來評估這個學生是不是適合某個科系,如何利用科技有效地從審查資料中找到合適的學生,將是我們面臨到的課題。

隨著人工智慧的普及,在人才招募分析(Recruitment Analysis)中,許多文獻利用人工智慧的技術來做履歷匹配、提取的應用,他們著重分析面試者的技能是否符合公司所需的技能、條件,近年來,隨著求職網站不斷蓬勃發展,在成千上萬的工作及履歷中,對於公司如何在這麼多的履歷中找到適合自己的員工,以及求職者如何在這麼多的工作中找到適合自己的,因此有許多人開始研究個人-工作配對問題(person-job fit),在相關研究中對於文本的表示,沒有考慮到不同用詞但代表相同意義的用語,如具有軟體開發經驗跟寫過android app,然而大部分的研究只針對簡歷中的一部分資料去配對如:工作經驗,無考慮到整份履歷中的其他資料,如:教育程度、技能及自傳等等的配對。大部分的研究都是簡歷與工作需求這樣短文本—短文本的配對,缺少長文本—短文本的配對研究。

在我們的研究中審查資料屬於半結構化的長文本資料,表格包含畢業學校、專長、技能、幹部經驗等等的資料,以及自傳、讀書計畫、申請動機等長文本資料,將審查資料轉換成好的表示將是一大挑戰,尤其長文本在轉換特徵表示很難捕捉到語義所以導致配對效果不好,我們提出的自動評估學生審查資料的方法,能夠進行長文本—短文本的配對,配對審查資料跟尺規。尺規是一個短文本,尺規是一個表格,而這個尺規是很廣泛且模糊的標準,所以我們要先對這個尺規建立一套方法,像是國際性比賽、有程式設計經驗且表現傑出,何謂國際性比賽、程式設計經驗,我們要如何去配對這些詞語,以便機器了解這個尺規並且能夠根據尺規去評分,所以我們運用建立知識圖的方法,替尺規去擴增豐富度。然而語言的表達形式很多,尺規中的要求實際上審查資料也滿足這個要求,但是表達方式或用詞用語不一樣,這時候透過知識圖得到得到表示會包含相關資訊,不但有利於配對,在消弭不同用語用詞也有幫助。

本研究設計了評估學生審查資料方法,利用配對審查資料與尺規去量化審查資料的分數,首先,對尺規資料去建立知識圖,將審查資料利用結合知識圖的文件編碼器(Document Encoder)進行編碼得到知識表示(Knowledge Representation),對於尺規中有程度上差別的評分如:傑出、優良、普通等等,制定一個方法給予審查資料尺規程度嵌入(Rule-based representation),接著,拼接文本表示及尺規程度嵌入得到增強表示(Enhanced Representation),將增強表示送到配對模型(Match Model)並與尺規配對得到最終的審查資料分數,透過機器自動對審查資料評分,能夠提供審查委員一個較客觀的參考。